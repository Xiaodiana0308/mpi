\documentclass[UTF8]{article}%z指定文档类型
\usepackage[UTF8]{ctex}%显示中文
\usepackage{graphicx}%引用图包
\usepackage{amsfonts,amsmath,amssymb,amstext,array}%数学相关宏包
\usepackage{xcolor}
\makeatletter
\renewcommand*\env@matrix[1][*\c@MaxMatrixCols c]{%
\hskip -\arraycolsep
\let\@ifnextchar\new@ifnextchar
\array{#1}}
\makeatother

\begin{document}%文章开始

\title{并行计算}%文章题目
\maketitle% 显示上述标题信息

\section{矩阵乘并行计算}

\subsection{矩阵基本性质}

\subsubsection{加法}

可交换顺序、可分配可结合、A+(-A)=O

\subsubsection{数乘}

可交换顺序、可分配可结合。

\subsubsection{乘法}

不可交换顺序、可分配可结合。

\subsubsection{转置}

$(A+B)^T=A^T+B^T$

$(kA)^T=kA^T$

$(AB)^T=B^{T}A^T$

$(AT)^T=A$

\subsubsection{共轭}

$(A)_{i,j}=\overline{A_{i,j}}$

矩阵内实部不变,虚部取负。

\subsubsection{注意}

$(A+B)(A+B)=A^2+AB+BA+B^2$

\subsubsection{相关定义}

\begin{itemize}
    \item 行列式:是一个函数,其定义域为det的矩阵A,取值为一个标量,写作det(A)或|A|,A为n×n的正方形矩阵。
    \item 余子式:n阶行列式D中,把元素$a_{oe}$所在的第o行和第e列划去后,留下来的n-1阶行列式叫做元素$a_{oe}$的余子式,记作$M_{oe}$。
    
    k阶余子式:行列式D中划去了k行k列,划去的交叉部分组成子式A(即元素),称剩下的为行列式D的k阶子式A的余子式。

    \item 代数余子式:将余子式$M_{oe}$再乘以-1的o+e次幂记为$A_{oe}$,$A_{oe}$叫做元素$a_{oe}$的代数余子式。
    
    同理k阶代数余子式。此时o、e为行和列的序号累加。
   
    \item 特征值:对于n阶方阵A,如果存在数m和非零n维列向量x,使得Ax=mx成立,则称m是A的一个特征值(characteristic value)或本征值(eigenvalue)。
    \item 特征向量:对于一个给定的线性变换,它的特征向量(本征向量或称正规正交向量)v满足经过该线性变换之后,得到的新向量仍然与原来的v保持在同一条直线上。
    \item 迹:n阶方阵A的对角元素之和称为矩阵A的迹(trace),记作tr(A)。
    \item 正定矩阵:n阶方阵A,如果对任何非零向量z,都有$z^TAz>0$,其中$z^T$表示z的转置,就称M为正定矩阵。大于等于0则为半正定矩阵,小于0则为负定矩阵。
\end{itemize}

\subsubsection{初等变换}

初等变换:设A是m×n矩阵,进行倍乘、互换、倍加行(列)变换,统称为初等变换。包括:

\begin{itemize}
    \item 倍乘:用非零常数k乘A的\textbf{某行(列)}的每个元素。
    \item 互换:互换A的某两行(列)的位置。
    \item 倍加行(列):将A的某行(列)元素的k 倍加到另一行(列)。
\end{itemize}

初等矩阵:单位矩阵经\textbf{一次}初等变换得到的矩阵称为初等矩阵。

等价矩阵:矩阵A经过有限次初等变换变成矩阵B,则称A与B等价(可能有多个矩阵与A等价,其中等价的最简矩阵被称为A的等价标准型)

性质:用初等矩阵P左乘(右乘)A,其结果PA(AP)相当于对A作相应的初等行(列)变换。

\subsection{传分块统方法}

对于矩阵a被分成2×2四块的情况:

$
\begin{bmatrix}
    a_{0,0} & a_{0,1}  \\
    a_{1,0} & a_{1,1}
\end{bmatrix}
$

有:$c_{0,0}=a_{0,0}*b_{0,0}+a_{0,1}*b_{1,0}$

该情况下,每个线程(计算块)中都存储一行A和一列B(矩阵块),(又要传递又要储存)大大增加了存储量,存储量由O(n平方)—>O(n立方)

\subsection{Cannon方法}

该算法在每次计算完成后让计算块内的数据有规律的传递移动,记为M×M矩阵,有:

1. A总体上子块从右往左循环移动1步;$a_{i,j}->a_{i,j-1}$

2. B总体上子块从下往上循环移动1步;$b_{i,j}->b_{i-1,j}$

3. 每个计算块正常相乘,并存入C块中,$c_{i,j}=c_{i,j}+c$;

4. 重复上述过程,累计M次;

最终得到相乘的结果。

\section{线性方程组的并行求解}

\subsection{直接求解法}

\subsubsection{LU分解算法}

对于矩阵形式的线性方程组:Ax=b,如果A满足为方阵且可逆,则A可以被分解为下三角矩阵 L(Lower Triangle Matrix)和上三角矩阵U(Uppder Triangle Matrix)的乘积。即:

PA = LU

得:LUx=Pb

则方程可以被分解为(L)y=(Pb)和(U)x=(y),通过两个相似的步骤依次求解y和x即可。

基本原理是利用高斯消元,原理是在求解方程组Ax=b时将系数矩阵A和右向量b组成增广矩阵,对其进行行的初等变换,最终得到上三角初等矩阵,则自下而上可求得各个x。

在不带入b的情况下,对A矩阵的初等变换操作可以以置换矩阵$E_{ij}$来表示,表示交换第i行和第j行。而此矩阵乘以一个置换矩阵P即可化为下三角形式。

\paragraph{置换矩阵P:}~{}

对于单位矩阵E,交换其内部的行列,即可得到置换矩阵,与矩阵相乘时,对单位矩阵的操作(交换、乘加、乘(但仅交换属于置换矩阵))都会作用到相乘的矩阵上。

\subsubsection{Gauss直接消去}

对矩阵:
$
\begin{bmatrix}
    a_{1,1} & a_{1,2} & ... & a_{1,n}\\
    a_{2,1} & a_{2,2} & ... & a_{2,n}\\
    ...     &                        \\
    a_{m,1} & a_{m,2} & ... & a_{m,n}\\
\end{bmatrix}
$

\begin{itemize}
    \item 从上往下,第二行减去乘以系数$a_{2,1}/a_{1,1}$的第一行:
    
    $a_{2,k}=a_{2,k}-a_{1,k}*a_{2,1}/a_{1,1}$

    更新第二行的值$a_{2,k}$。

    \item 第三行减去乘以系数$a_{3,1}/a_{1,1}$的第一行,减去乘以系数$a_{3,2}/a_{2,2}$的第二行:
    
    $a_{3,k}=a_{3,k}-a_{1,k}*a_{2,1}/a_{1,1}$
    
    $a_{3,k}=a_{3,k}-a_{2,k}*a_{3,2}/a_{2,2}$
    \item 对于第i行的处理,第j列元素有:
    
    \textcolor{red}{$a_{i,j}=a_{i,j}-\sum_{k=1}^{k<i}{(a_{k,j}*a_{i,k}/a_{k,k})}$}  

    第k行乘的因子始终为:$a_{i,k}/a_{k,k}$

    \item 最终得到上三角初等矩阵U。L和P通过反推变换过程可以得到。
    \item 对向量b做相同变换,则可从下而上,逐渐求解出各个未知数。
\end{itemize}

\paragraph{列主元Gauss消去方法:}~{}

\begin{itemize}
    \item 类似于直接方法,但避免了$a_{j,j}$为0时无法消元的情况。将从第j列的$a_{j,j}$及其以下的各元素中选取绝对值最大的元素,然后通过行变换将它交换到主元素$a_{j,j}$的位置上,再进行消元。
    \item 对于第j列,首先找到该列中最大值的所在行l(\textcolor{red}{$l\geq j$即可}),记最大值为$a_{l,j}$。
    \item 然后将第j列第j行即主元$a_{j,j}$的值,与$a_{l,j}$进行对比:
    
    如果$a_{l,j}$等于0就直接退出(该矩阵无法求解);
    
    如果不相等,则将第j行与第l行进行交换,使第j列最大值在主元位置: $swap(a_{l,j},a_{j,j})$。

    \item 对于\textcolor{red}{$i>j$}的每一行i(k表示列),执行操作$a_{ik}=a_{ik}-a_{ik}\times a_{ij}/a_{jj}$,注意j始终指当前的第j列。
    \item 然后依次处理\textcolor{red}{$0\leq j < n$}各列。
\end{itemize}

\subsubsection{Gauss消去并行计算方法}

\begin{itemize}
    \item 并行部分以列为块分割,传递的是因子$f_{i-j}=a_{i,j}/a_{j,j}$(使相邻块加乘相同,表示\textcolor{red}{第i行用的(减去)第j行乘的因子});以及为了定位行的交换,还需要传递最大值所在行l。
    \item 并行部分只计算落在当前线程列区间内的列的因子,并计算区间内列的交换和加乘。
    
    设并行区间为[j0,j1]、行列式为m*m大小。
    
    如果列在区间内:直接计算因子$f_{i-j}$,并向下一个线程发送因子($i\subseteq [j0,j1],j\subseteq [j0+1/j1+1,m]$),(因子\textbf{包括之前线程传来}的,反正传总的就行,没写的也是0)和l。
    
    如果列在区间外(指小于区间),就接受对应列的因子和l。大于也可,反正都是0。

    最终得到区间内需要的所有的因子和l。

    \item 利用因子和l,先按顺序进行所有的列交换,然后按顺序对行加乘运算。最终得到上三角矩阵。
\end{itemize}

\paragraph{三角矩阵的并行求解:}~{}

以下三角矩阵为例:步骤:

$
\begin{bmatrix}
    a_{1,1} & 0       & 0       & ... & 0      \\
    a_{2,1} & a_{2,2} & 0       & ... & 0      \\
    a_{3,1} & a_{2,3} & a_{3,3} & ... & 0      \\
    ...                                        \\
    a_{m,1} & a_{m,2} & a_{m,3} & ... & a_{m,n}\\
\end{bmatrix}
$

% 假设P=3,有:
% 3 i=3   j=0 
% 4       j=1   i+j    i+p-2
% 5       j=2   i+j    i+p-1
% 6 i=6   j=0
% 7       j=1
% 8       j=2
% 9 i=7   j=0

% 11
% 21  22
% 31  32  33
% 41  42  43  44
% 51  52  53  54  55
% 61  62  63  64  65  66
% 71  72  73  74  75  76  77

\begin{itemize}
    \item 并行的部分为列,但出于处理器负载均衡的考虑,每一个并行块并不是相邻的若干列,而是分开的,类似于123,123,123这样的交替排列,每一组大小为P列(等于并行的数量),称为卷帘。
    \item 并行部分原理:从k=0列开始:
    \item 如果属于第一个线程,就使$u_i=b_i$(行$i\subseteq [0,n-1]$),并使$v_i=0$(行$i\subseteq [0,p-2]$)。
    
    否则使$u_i=0$,行范围同上。
    
    (即只在第一个线程处对u赋右值、对v赋初值0)

    \item 判断属于第myid线程,i从myid开始增加,对每一个线程内的第i行(0到p-1),\textcolor{red}{i以p为步长递增},直到最后n:
    
    $for \quad i=myid \quad step \quad p \quad to \quad n-1 $

    \item 在i=0的情况下,不接受数据;否则接收n维向量$v_{recv}$。
    \item 计算$x_k=(u_i+v_0)/a_{ik}$。
    
    \textbf{整个0到p-1只有这一行即k行的x是求的,剩下的由接下来的线程求。}

    \item 更新传入的v向量:
    
    $v_j=v_{j+1}+u_{i+j+1}-a_{i+j+1}*x_k$ , j=0,...,p-3

    $v_{p-2}=u_{i+p-1}-a_{i+p-1}*x_k$ , 类似于j=p-2(改成p-1?)

    \textbf{j在这里指的是每一个i下,对应的行数,随i的变化v向量不断更新。}

    \item 向下一线程发送v向量$v_{send}$。
    \item 更新[i+p,u-1]行范围内的u向量:
    
    $u_j=u_j-a_{jk}*x_k$ , j=i+p,...,n-1

    \item k=K+1。
    \item 继续循环i,最终完成全部求解。\textcolor{red}{每次循环只处理一列!!}
    \item 注意:{
        \begin{itemize}
            \item \textcolor{red}{v用于将本线程的计算结果对接下来的方程的影响传递给其他线程。大小取决于线程总数目。每次循环时都会变化,由新计算出的解来更新。}
            \item 其每次循环计算P次,恰好到下一个本线程之前。
            \item 注意在v的更新中,更新后的$v_0$指的是当前的k对应的k+1行,即迭代更新。但注意其中u(old)的更新同样采用了迭代的方法,在继承下一行$v_j$的同时,有引入了当前线程的u,而最下行的v则\textcolor{red}{并没有添加新的u},也就是说\textcolor{red}{在下次循环到本线程时,v向量所有的u都不是本线程的了},因此在下一次循环到本线程时,计算x时仍然需要加u。(但这样设计是因为v大小只为一个循环的,只能考虑这么多的$a_i*x_i$计算,到下一个循环时需要重新计算$a_{i+p}*x_{i+p}$)
            只含有当前线程计算的全部结果对方程的影响。因此在下一个线程接收到这个数据时,\textcolor{red}{并不包括本线程之前的所得解的影响},因此在x计算公式中可以看到,加上了本线程储存的u(new,用上次计算出的x更新过)。
            \item \textcolor{red}{u表示未计算的所有行方程$d+bx=c-a$中的c-a项,每次计算出一个$x_{k-1}$值后,就会对u进行更新,未解行i的方程(之后的方程)全部减去$x_{k-1}*a_{ik}$}。
            \item u在初始情况下即第一个线程时被初始赋值为b,即为方程右值。
            \item \textcolor{red}{u仅用于标记线程内部所得解对c-a的影响,外部影响必须全部通过传递的v得到。}
        \end{itemize}
    }
\end{itemize}

\subsection{迭代解法}

可以对每一行的方程迭代部分进行划分,每一个计算块处理若干行的迭代方程。

\section{FFT并行算法}

\subsection{复数基本知识}

\begin{itemize}
    \item 复数乘法的在复平面中表现为辐角相加,模长相乘;
    
    即 $(a_1,\theta_1)*(a_2,\theta_2)=(a_1*a_2,\theta_1+\theta_2)$

    \item 单位根:复数w满足$w^n=1$,称为n次单位根。如图所示:
    
    {
        \begin{figure}[htb!]%插入图片
            \includegraphics[width=0.4\textwidth]{3-1.png}
        \end{figure}
    }
    
    总n次第m个根记为$w_n^m$,其中n为2的整数倍,则满足性质:

    $w_n^m=-w_n^{m+n/2}$

\end{itemize}

\subsection{快速傅氏变换FFT原理}

\subsubsection{物理意义}

\paragraph{傅里叶变换:}~{}

对于周期函数,是将$f(t)$分解为无数个\textbf{不同频率、不同幅值}的正、余弦信号。用频谱函数表示,自变量是频率$\omega $,因变量是幅值。函数是离散的,自变量都是基频$\omega_0$的整数倍。

对于非周期函数,则是求频谱密度函数,自变量是$\omega $ ,因变量是信号幅值在频域中的分布密度,即单位频率信号的强度。

\textbf{可以将频谱函数和频谱密度函数类比为离散概率分布和概率密度函数。}

\paragraph{快速傅氏变换:}~{}

是离散傅氏变换的快速算法,是对离散傅立叶变换的改进。可用于加速多项式的乘法,将复杂度从$\varTheta(n^2)$优化为$\varTheta(n \log n)$。

\subsubsection{DFT}

对于连续的傅里叶变换,已知:

$F(f)=\int_{-\infty}^{+\infty} f(t) e^{-j 2 \pi f t} d t$

其目的是得到信号的频谱密度函数(t->w(f)),DFT就是t和f都为离散版的傅里叶变换。

由于计算机也只可能计算出有限个频率上对应的幅值密度,因此最终也需要转为离散的情况。转化步骤:

\begin{itemize}
    \item 采样:
    
    利用狄拉克函数的性质:

    $\int_{-\infty}^{\infty} \delta (t-t_0)f(t) \,dx=f(t_0)$

    能够筛选出f(t)在$t_0$时刻的函数值$f(t)$),从而采样$t_0$点,记采样周期为$T_s$,并认为采样附近函数值相等,则f(t)可近似为:

    $f_s=\sum_{n=-\infty}^{\infty}f(t)\delta (t-nT_s) $

    \item 时域离散化:
    
    对采样结果进行傅里叶变换,使时域为无限大求和形式,即:

    $F(\omega )=\sum_{n=-\infty}^{\infty}f(nT_s)e^{-j\omega nT_s}$
    
    \textbf{即每一个$\omega $点的值$F(\omega)$都是由无数个取样点的和组成,且只能得到指定位置的点的值($k\omega$)。}

    \item 频域离散化:
    
    选取有限的N个时刻T,采样间隔同为$T_s$,求和只求范围内的,则可求的k个点的F值为:
    
    $F[k]=\frac{1}{N} \sum_{n=0}^{N-1} f[n] e^{-j \frac{2 \pi}{N} k n}$\qquad k=0,1,2,...,N-1(表示取样点)

    \textcolor{red}{其中$F[k]=F(k\omega_0)T_s=F(\omega_k)T_s$表示频域,}
    
    \textcolor{red}{$f[n]=f(nT_s)=f(t_n)$表示时域。}

    即得到频谱函数$F[k]$,表示的是$k\omega_0$时刻信号幅值大小。

    \item 替代:取样点总数N->n,当前取样点数n->j,复数标志j->i。得:

    $F[k]=\frac{1}{n} \sum_{j=0}^{n-1} f[j] e^{- \frac{2 \pi i j k}{n}}$\qquad k=0,1,2,...,n-1
    
    记\textcolor{red}{$y_k=F(\omega_k)*nT_s$,$x_j=f(t_j)$},则上式可以化为:

    $y_k=\sum_{j=0}^{n-1} x_j e^{- \frac{2 \pi i j k }{n}}$\qquad k=0,1,...,n-1

    \textcolor{red}{k是大循环,j是小循环。}

    \item 注意:该方法的时间复杂度为$\varTheta(n^2)$。
\end{itemize}

\subsubsection{FFT}

用于在DFT的基础上,减少其复杂度。

基本原理:

\begin{itemize}
    \item 记$\omega (n)=e^{-\frac{2 \pi i}{n}}$,则$\omega (n)^k$为方程$x^n=1$的第k根。上式化为:
    
    $y_k=\sum_{j=0}^{n-1} x_j \omega (n)^{kj}$\qquad k=0,1,...,n-1

    \item 同样有性质成立:
    
    性质1:$\omega (n)^{2k}=\omega (n/2)^k$ 不同于:\textcolor{red}{$[\omega (n)^{k}]^2=\omega (2n)^k$}

    性质2:$\omega (n)^{kn}=1$

    性质3:$\omega (n)^{kn/2}=-1$
    
    性质4:$\omega (n)^{k}=\omega (n)^{k+n}=-\omega (n)^{k+n/2}$

    \item 则可利用这些性质化简DFT方程。
    
    把$\omega (n)^{j}$视为整体,首先考虑单个方程的化简。(\textcolor{red}{化简内循环j})
    
    将其拆分成奇数和偶数两部分相加,有:

    $y_{k}=\sum_{j=0}^{n/2-1} x_{2j} \omega(n)^{2jk}+\sum_{j=0}^{n/2-1} x_{2j+1} \omega(n)^{(2j+1)k}$

    记n=2m,利用性质1和4,化简为:

    \textcolor{red}{$y_{k}=\sum_{j=0}^{m-1} x_{2j} \omega(m)^{jk}+\omega(n)^{k}\sum_{j=0}^{m-1} x_{2j+1} \omega(m)^{jk}$}
    
    \item 考虑方程间的化简:(\textcolor{red}{化简大循环k})
    
    由性质4:
    
    $(\omega (m)^{k+m})^j=\omega (m)^{kj}$、$(\omega (n)^{k+m})^j=(\omega (n)^{k+n/2})^j=-\omega (n)^{kj}$

    因此可得$y_{k+m}$的表达式与$y_k$几乎一样,区别仅在于第二部分的$\omega(n)^{k+m}$由于对应方程级数仍然为n,因此变化为$-\omega(n)^{k}$。
    
    最终得到:
    
    $\left\{\begin{array}{l}
        y_{k}=\sum_{j=0}^{m-1} x_{2 j} \omega(m)^{k j}+\omega(n)^{k} \sum_{j=0}^{m-1} x_{2 j+1} \omega(m)^{k j} \\ \\
        y_{k+m}=\sum_{j=0}^{m-1} x_{2 j} \omega(m)^{k j}-\omega(n)^{k} \sum_{j=0}^{m-1} x_{2 j+1} \omega(m)^{k j} \\ \\
        k=0,1, \ldots, m-1
    \end{array}\right.$

    可记为:

    $\left\{\begin{array}{l}
    y_k=G(x^2)+xH(x^2) \\ 
    y_{k+m}=G(x^2)-xH(x^2)
    \end{array}\right.$

    \textcolor{red}{注意其中的x实际上指的是$\omega(m)$,而非前式的x。}
    
    也就是说,只要能够得到$y_k$,就一定能够得到$y_{k+m}$。因为每一个m、k下,H和G总是相等的。

    \item 分治方法:
    
    完整的分治过程\textcolor{red}{不仅包括利用k+m与k的关系不断对方程组进行减半的拆分,还包括对方程内的不同指数的项按奇偶进行拆分}。
    
    对于n=2m的划分可以一直进行下去,但由于每一次方程数目减半,方程内也需要继续进行划分以减少系数,同时每一行y的表达式增加,直到最简单的形式:\textcolor{red}{H和G中不含有x即$y=G+xH$}。
    
    使n=n/2,m=n/2。只对$y_k$处理,然后对G和H分别建立方程$y_{k'}=G(x^2),\quad y_{k''}=H(x^2)$,有$y_{k'}+y_{k''}=y_k$。\textcolor{red}{由于G和H在形式上是完全一样的},因此处理步骤相同。以$y_{k'}$为例,使用$x$替代$x^2$(利用性质1恰好使上一步的$\omega(m)$中的m减小一半,对应新的m),然后就可以按照前文的步骤,将奇部取出x,即$y_{k'}=G'(x^2)+xH'(x^2)$。然后以此类推。

    \item 复杂度:由于对于个点n而言,一共需要在$n^{0.5}$个位置建立方程求解(m对应的位置才需要),而每一次需要进行n次乘法(x与$\omega$相乘),因此总的复杂度量级为$nlog_2(n)$。
    \item 注意:方程数目或多项式系数+1必须为$2^n$次方,否则需要补零。
    \item 注意:可见求解中需要计算全部的$\omega(m)^{kj}$,m=2,4,...,n/2、k=0,1,...,n-1。但利用性质1,k计算到n/2-1就可以了。
\end{itemize}

\subsection{多项式乘法与FFT}

\subsubsection{多项式的表示方法}

系数表示法:用一个多项式的各个项系数来表达该多项式。

点值表示法:把n-1阶多项式看成一个函数,从上面选取n个点,从而利用这n个点来唯一的表示这个函数。每个点记为$(x_i,y(x_i))$。

DFT:多项式由系数表示法转为点值表示法的过程;

IDFT:把一个多项式的点值表示法转化为系数表示法的过程。

\textcolor{red}{FFT就是通过取某些特殊的x的点值来加速DFT和IDFT的过程。}

\subsubsection{点值表示法与FTT关系}

在点值表示法下,单纯的多项式相乘复杂度为$n$,因为在向量乘中,$x_k$保持不变,而仅仅需要将各项$f(x_i)$和$g(x_i)$相乘。

对于DFT和IDFT过程,复杂度则取决于这两个转化过程:$y_i=a_0+a_1*x_i+a_2*x_i^2+...+a_n*x_i^n$方程组,已知A和X向量,求解Y;已知Y和X向量,求解系数向量A。最适合带入的X的值即为方程$x^n=1$的根,$\omega_n^k、k=0,1,..,n-1$。由于每个方程系数是一样的,这样就可以利用之前复数的周期性质,减少乘的数量,快速转化。复杂度同FFT算法,为$nlog_2n$量级。

带入x,对于第k行方程,表示为:

$y_k=\sum_{j = 0}^{n-1} a_jx_k^j = \sum_{j = 0}^{n-1} a_j(\omega_n^k)^j$\qquad k=0,1,...,n-1

参考FFT基本式: $y_k=\sum_{j=0}^{n-1} x_j \omega (n)^{kj}$\qquad k=0,1,...,n-1

则若视$a_j$为$x_j$,则两式完全相等。可以采用同样的方法进行化简。


编写程序时注意($e^{\theta i}=cos(\theta)+isin(\theta)$):对于$\omega_n^k=e^{-{2\pi i k}/n}$,在传统意义上表示时域向频域的转化关系,但在多项式中则表示原项乘以了逆矩阵且扩大了n倍。\textcolor{red}{多项式乘法中,如果只是单纯的相乘,应采用$e^{{2\pi i k}/n}$。}

\subsubsection{FFT加速DFT}

在DFT中,a与x已知,求解y;

\paragraph{基本FFT实现:}~{}

\begin{itemize}
    \item 待乘式次数补0,使满足$n=2^l$;
    \item 计算出所有的x:$\omega(m)^k$、$m=2,4,...,n/2$、$k=0,1,...,n/2-1$。
    \item 利用FFT部分的奇偶分治,只考虑第0行方程,将其拆分到最终只剩两个与$\omega$无关的参数$a$:
    
    即n=1时:$G_0=y_0^{(0)}=a_0*\omega_1^0=a_0$;$H_0=y_1^{(0)}=a_1*\omega_1^0=a_1$。

    \item 然后在纵向和横向上不断“滚雪球”一般累加回各项或方程,顺序与拆分时相反。
    
    {
        \begin{itemize}
            \item 首先考虑横向的累加:
            
            每轮累加时都满足:\textcolor{red}{下一个偶数项/奇数项=上一个偶数项+$\omega_n^k×$上一个奇数项}。其中n是当前方程下的n。
            
            则通过这样的步骤以翻倍的速度不断加回之前被分治的各奇偶项,直到得到最终的y。由于H和G都是一轮一轮累加出来的,避免了利用求和公式累加幂导致j对$\omega$影响。

            \item 然后考虑纵向的累加:
            
            在每一轮横向累加时利用当前n值下的周期性,每一个n下增加对$k+n/2$列的计算即可(即蝴蝶操作)。但注意由于每一次的原方程并不完整,因此每一次纵向回滚时都需要重新计算所有的行方程(更新y)(就是用新的G和H变换加减号来计算)。

        \end{itemize}
    }            
    
    \item 示例:
    
    以8项式为例,首先考虑横向的分治和回滚,略去行系数k:
    
    $x_0$、$x_1$、$x_2$、$x_3$、$x_4$、$x_5$、$x_6$、$x_7$ \qquad n=8
    
    分治步骤:

    第一次:$x_0$、$x_2$、$x_4$、$x_6$;$x_1$、$x_3$、$x_5$、$x_7$ \qquad n=4

    第二次:$x_0$、$x_4$;$x_2$、$x_6$;$x_1$、$x_5$;$x_3$、$x_7$ \qquad n=2

    第三次:$x_0$;$x_4$;$x_2$;$x_6$;$x_1$;$x_5$;$x_3$;$x_7$ \qquad n=1

    回滚:

    最开始:$G_0=a_0$;$H_4=a_4$;$G_2=a_2$;$H_6=a_6$;$G_1=a_1$;$H_5=a_5$;$G_3=a_3$;$H_7=a_7$;\qquad n=1

    滚1次:$G_{04}=G_0+\omega(n)H_4$;$H_{26}=G_2+\omega(n)H_6$;$G_{15}=G_1+\omega(n)H_5$;$H_{37}=G_3+\omega(n)H_7$;\qquad n=2

    滚2次:$G_{0426}=G_{04}+\omega(n)H_{26}$;$G_{1537}=G_{15}+\omega(n)H_{37}$\qquad n=4

    滚3次:$Y=G_{0426}+\omega(n)H_{1537}$\qquad n=8

    得到了最终的y值。

    然后考虑纵向的回滚,记一开始为第0行:
    
    第一次:0->1\qquad n=2,m=1

    第二次:0->2,1->3\qquad n=4,m=2

    第三次:0->4,1->5,2->6,3->7 \qquad n=8,m=4

    最终求得了每一行的y值。

    注意:\textcolor{red}{除非最后一次(n)计算,其他次(n)的计算都是不完整的,因此每一个n下都需要重新计算所有的其余列。}

    \item 递归程序参考:
    
    {
        \begin{figure}[htb!]%插入图片
        \includegraphics[width=0.5\textwidth]{3-2.png}
        \end{figure}
    }
    
    \item 此过程复杂度为$nlog_2n$。
\end{itemize}

\paragraph{高效FFT实现:}~{}

之前的FFT实现中,在行之间的计算顺序每次(n)都是按0到n增加的,但是行内则是对每一项进行了重新排列,在递归方法下需要花费大量空间用于创建和维护数组。而如果一开始每一行的项就是已经是排列后的,则可以利用迭代法来求解,提高FFT效率。

重要规律:\textcolor{red}{在原始的顺序下,每个项序号用二进制表示(四位),然后把每个数的二进制顺序翻转一下,就是最终拆分完全后每个数的序号。}

蝴蝶变换:输入$x_1$、$x_2$,通过$y_1=x_1+x_2$、$y_2=x_1-x_2$,使最终输出$x_1=y_1$、$x_2=y_2$的方法。

迭代法FFT实现:

\begin{itemize}
    \item 翻转多项式所有的系数$a_i$,变化为需要的排列顺序;
    \item \textcolor{red}{以a的次序为处理顺序;}
    \item 进行主循环,记step,从1开始每次自身乘2递增直到n-1;
    
    (记step个系数的H+xG的值为大单元,则step表示分治下各部分系数数量为step的情况(不管行))
    
    \item 计算当前step下的$\omega_n^1=e^{{2\pi}/n i}$;\textcolor{red}{(由于因为这是逆操作,step始终是只为原来一半的,因此利用当前的H+xG求解新H或G时,在x中需要将step乘2,即$e^{\pi /n i}$)}
    \item 进行中循环,记j,从0开始每次增加2倍的step直到n-1;\textcolor{red}{(将向量a按当前step大小全分割(总计n/step个),每一次循环处理两个大单元(序号间隔step),最终得到全部更新的a向量)}
    
    (j表示当前处理的a向量范围$[j,j+2step)$))

    可以视为对单行方程的分割。随step增加j取值不断减少。当step=n/2时,不分割,对应的$a_j$即j行的计算值。

    \item 进行小循环,记k,从j开始+1增加step个数为止;\textcolor{red}{(利用之前step下计算的上一轮的a向量值,来得到可求的每一行方程内对应传入的step位置的G+xH的值,即更新一部分a向量(2step个))}
    
    (k即表示a向量的序号,每次小循环会更新j即2step的部分a向量,直到全部更新完成)\textcolor{red}{(在第一次传入j中表示行的序号,从0到step;而在之后传入的j下,a向量的序号并不对应于列,需要减去之前的序号(即$k-j$或$k-2*step*l$才表示列))}

    可以视为单方程内分割下的方程间分割的分别计算。随step增加k取值不断增加,当step=n/2时,计算量最大,恰好为全部的a数。

    \item 注意:传入的a矩阵为按顺序排列的系数向量,由于采用了重复赋值更新,在一轮大循环后,a的意义就已经发生变化了,a向量按一定的规律在不同的step下排列,但最后会表示为每一行的累加值。
    \item 程序:
    
    {
        {
        \begin{figure}[htb!]%插入图片
        \includegraphics[width=1.0\textwidth]{3-4.png}
        \end{figure}
        }
    }

    \textcolor{red}{注意需要考虑不同行中k的影响,这是通过小循环中从第0行开始,每次循环wnk项乘以一个wn来的($\omega_{2n}^{0+1+1+\dots}$)。}

    \item 蝴蝶变换示意图:
    
    {
        {
        \begin{figure}[htb!]%插入图片
        \includegraphics[width=1.0\textwidth]{3-3.png}
        \end{figure}
        }
    }

\end{itemize}

\subsubsection{FFT加速IDFT}

在IDFT中,y与x已知,求解a。

将方程写为如下矩阵形式:(之前也是这种形式,只不过a向量乘进去了)

$\left[\begin{array}{c}y_{0} \\ y_{1} \\ y_{2} \\ y_{3} \\ \vdots \\ y_{n-1}\end{array}\right]=\left[\begin{array}{cccccc}1 & 1 & 1 & 1 & \cdots & 1 \\ 1 & \omega_{n} & \omega_{n}^{2} & \omega_{n}^{3} & \cdots & \omega_{n}^{n-1} \\ 1 & \omega_{n}^{2} & \omega_{n}^{4} & \omega_{n}^{6} & \cdots & \omega_{n}^{2(n-1)} \\ 1 & \omega_{n}^{3} & \omega_{n}^{6} & \omega_{n}^{9} & \cdots & \omega_{n}^{3(n-1)} \\ \vdots & \vdots & \vdots & \vdots & \ddots & \vdots \\ 1 & \omega_{n}^{n-1} & \omega_{n}^{2(n-1)} & \omega_{n}^{3(n-1)} & \cdots & \omega_{n}^{(n-1)^2}\end{array}\right]\left[\begin{array}{c}a_{0} \\ a_{1} \\ a_{2} \\ a_{3} \\ \vdots \\ a_{n-1}\end{array}\right]$

需要在等式两边左侧乘以$\omega$的逆矩阵,即可变为$A=\omega^{-1}y$的格式,与之前的y=ax形式完全一样,可用相同方法求解。即输入为y向量,返回的为新的系数向量a。

可以证明,\textcolor{red}{对于矩阵每一项取倒数再除以n就是该矩阵的逆矩阵}。

注意为保证输出a向量的顺序,也需要在计算前对y向量进行翻转。

则程序基本上与DFT过程相同,\textcolor{red}{区别在于对系数的处理中,将y向量每一项除以n;然后在H+xG处理中,x的初始定义由$\omega_n^k=e^{{2\pi i k}/n}$变为$\omega_n^k=e^{-{2\pi i k}/n}$(e的系数加个负号)}。

完整的程序可以写为一个函数,除上述操作外其他部分完全一样。 

\subsection{二维串行FFT算法}

可以由两个方向的一维FFT来完成。

\section{MPI并行程序设计基础}

\paragraph{与pthread区别}~{}

定义:Massage Passing Interface:是消息传递函数库的标准规范。是一种新的库描述,不是一种语言。

mpi是基于分布式内存系统,而openmp和pthread基于共享内存系统;

即mpi之间的数据共享需要通过消息传递,因为mpi同步的程序属于不同的进程,甚至不同的主机上的不同进程。相反由于openmp和pthread共享内存,不同线程之间的数据就无须传递,直接传送指针就行。

同时mpi不同主机之间的进程协调工作需要安装mpi软件(例如mpich)来完成。


\subsection{并行相关分类}

\paragraph{计算机架构:}~{}

\begin{itemize}
    \item SMP:SMP是对称多处理技术。具有多个CPU,所有的CPU共享一个内存,使用相同的地址空间。所有的CPU通过一条总线(bus)和内存以及IO设备(硬盘等)连接。总线同一时刻只能处理一个请求,当有多个CPU的访存访问请求时,只能一个一个处理。
    \item MMP:类似于集群,但MMP使用了更多定制化的组件,包括网络、处理器、操作系统等;而cluster运行通用操作系统,互连网络使用商业标准的IB和以太网设备连接,存储为SAN、NAS和并行文件系统。
    \item Cluster:集群,它至少将两个系统连接到一起,使两台服务器能够像一台机器那样工作或者看起来好像一台机器。基本特征是具备多个 CPU 模块,每一个CPU模块由多个CPU组成,而且具备独立的本地内存、 I/O 槽口等。
    \item MMP与Cluster区别:
    
    {
        \begin{itemize}
            \item MPP实际上是一台机器,这台机器有使用高速网络紧密连接的成千上万个处理器,只有一个操作系统。
            \item cluster实际上是有多台机器,每个机器有自己的操作系统(一般都是一样的)、硬盘、内存等,这些机器使用一些普通网络的一些变体连接起来,使用某些系统帮助分配任务给这些主机。
        \end{itemize}
    }
\end{itemize}

\paragraph{并行计算机系统结构编程模型(Flynn分类法):}~{}

\begin{itemize}
    \item 单指令单数据(SISD): SISD是标准意义上的串行机,具有如下特点:1)单指令:在每一个时钟周期内,CPU只能执行一个指令流;2)单数据:在每一个时钟周期内,输入设备只能输入一个数据流;3)执行结果是确定的。这是最古老的一种计算机类型。
    \item 单指令多数据(SIMD): SIMD属于一种类型的并行计算机,具有如下特点:1)单指令:所有处理单元在任何一个时钟周期内都执行同一条指令;2)多数据:每个处理单元可以处理不同的数据元素;3)非常适合于处理高度有序的任务,例如图形/图像处理;4)同步(锁步)及确定性执行;5)两个主要类型:处理器阵列和矢量管道。
    \item 多指令单数据(MISD):MISD属于一种类型的并行计算机,具有如下特点:1)多指令:不同的处理单元可以独立地执行不同的指令流;2)单数据:不同的处理单元接收的是同一单数据流。这种架构理论上是有的,但是工业实践中这种机型非常少。
    \item 多指令多数据(MIMD): MIMD属于最常见的一种类型的并行计算机,具有如下特点:1)多指令:不同的处理器可以在同一时刻处理不同的指令流;2)多数据:不同的处理器可以在同一时刻处理不同的数据;3)执行可以是同步的,也可以是异步的,可以是确定性的,也可以是不确定性的。这是目前主流的计算机架构类型。
\end{itemize}

\paragraph{并行程序类型:}~{}

\begin{itemize}
    \item 主从式M-S:即Master/Slaver模式。核心思想是基于分而治之,将一个原始任务分解为若干个语义等同的子任务,并由专门的工作者线程来并行执行这些任务,原始任务的结果是通过整合各个子任务的处理结果形成的。各子任务互不相干。
    \item 对称式SPMD:(Single Program Multiple Data)指单程序多数据。类似于SIMD,但在SPMD中,虽然各处理器并行地执行同一个程序,但所操作的数据不一定相同(即各处理器只在需要时进行同步,而不是同步地执行每一条指令)。
    \item 自主式MPMD:(Single Program Multiple)指多程序多数据。相比于SPMD,相当于各自进程执行各自的程序。SPMD和MPMD的表达能力是相同的,只是针对不同的问题编写难易而已。MPI是可以写SPMD和MPMD的并行程序的。
\end{itemize}

重点在SPMD。

\subsection{并行程序基本结构}

\begin{itemize}
    \item 进入MPI环境。产生通讯子(进程序号、进程数)。
    \item 程序主体。
    \item 退出MPI环境。
\end{itemize}

\subsection{MPI数据类型}

\begin{tabular}{l c c c r}
    MPI 数据类型                 &	C中对应数据类型           \\
    MPI\_SHORT                  &	short int               \\
    MPI\_INT	                &   int                     \\
    MPI\_LONG	                &   long int                \\
    MPI\_LONG\_LONG	            &   long long int           \\
    MPI\_UNSIGNED\_CHAR         &	unsigned char           \\
    MPI\_UNSIGNED\_SHORT	    &   unsigned short int      \\
    MPI\_UNSIGNED	            &   unsigned int            \\
    MPI\_UNSIGNED\_LONG	        &   unsigned long int       \\
    MPI\_UNSIGNED\_LONG\_LONG	&   unsigned long long int  \\
    MPI\_FLOAT	                &   float                   \\
    MPI\_DOUBLE	                &   double                  \\
    MPI\_LONG\_DOUBLE	        &   long double             \\
    MPI\_BYTE	                &   char                    \\
\end{tabular}


\subsection{MPI通信子(通信域)}

功能:MPI的通信在通信域的控制和维护下进行 → 所有MPI通信任务都直接或间接用到通信域这一参数 → 对通信域的重组和划分可以方便实现任务的划分

内容:
\begin{itemize}
    \item 上下文(context):提供了一个相对独立的通信区域,不同的信息在不同的上下文中传递,不同的上下文的信息互不干扰,上下文可以区分不同的通信。
    \item 进程组(group):组是一个进程的有序集合,在实现中可以看作是进程标识符的一个有序集。一个通信域对应一个进程组。
组内的每个进程与一个整数rank相联系,称为序列号,从0开始并且是连续的。
    \item 虚拟处理器拓扑(topology):\dots
\end{itemize}

附注:进程:一个进程对应一个pid号,同一个进程可以属于多个进程组(每个进程在不同进程组中有个各自的rank号),因此也可以属于不同的通信域。

默认(最大范围):MPI\_COMM\_WOLRD,这是MPI已经预定义好的通信子,是一个包含所有进程的通信子。\textcolor{red}{最大集}。

\textbf{通信域产生方法:}

\begin{itemize}
    \item 在已有通信域基础上划分获得:MPI\_Comm\_split
    \item 在已有通信域基础上复制获得:MPI\_Comm\_dup
    \item 在已有进程组的基础上创建获得:MPI\_Comm\_Create
\end{itemize}

\textbf{进程组产生方法:}

可以当成一个集合的概念,可以通过“子、交、并、补”各种方法。所有进程组产生的方法都可以套到集合的各种运算。

\subsection{MPI基本函数}

\subsubsection{并行环境管理函数}

\paragraph{MPI\_Init(\&argc, \&argv)}~{}

\begin{itemize}
    \item 功能:初始化MPI环境。产生一个通信子(称MPI\_COMM\_WORLD)
    \item 参数:
    {
        \begin{itemize}
            \item 就是C++main函数传入的参数,形式如上。
        \end{itemize}
    }
    \item 备注:必须保证程序中第一个调用的MPI函数是这个函数。不关心返回值。
\end{itemize}

\paragraph{MPI\_Finalize()}~{}

\begin{itemize}
    \item 功能:结束MPI环境。
    \item 参数:无
    \item 备注:任何MPI程序结束时,都需要调用该函数。不关心返回值。
\end{itemize}

\subsubsection{MPI通信子操作函数}

\paragraph{MPI\_Comm\_rank函数}~{}

int MPI\_Comm\_rank(

    \qquad MPI\_Comm comm, //[传入]当前进程所在的通信子

    \qquad int *rank //[传出]进程号

    ) 

\begin{itemize}
    \item 功能:获得当前进程的进程标识(进程号)。
    \item 返回值:不关心。
    \item 备注:在调用该函数时,需要先定义一个整型变量如myid,不需要赋值。将该变量传入函数中,会将该进程号存入myid变量中并返回。
\end{itemize}

\paragraph{MPI\_Comm\_size函数}~{}

int MPI\_Comm\_size(
    
    \qquad MPI\_Comm comm, //[传入](不一定本进程的)通信子。如果通信子为MP\_Comm\_WORLD,即获取总进程数

    \qquad int *size //[传出]进程数目
    
    ) 

\begin{itemize}
    \item 功能:是获取该通信子内的总进程数。
    \item 返回值:不关心。
    \item 备注:用法类似前一个。
\end{itemize}

\paragraph{MPI\_Comm\_dup函数}~{}

int MPI\_Comm\_dup(
    
    \qquad MPI\_Comm comm,//[传入]要复制的通信子

    \qquad MPI\_Comm *newcomm //[传出]新的通信子,具有相同的组和从源复制的任何缓存信息,但它包含新的上下文信息
    
    )

\begin{itemize}
    \item 功能:复制现有通信子及其所有缓存的信息
    \item 返回值:不关心。
    \item 备注:无。
\end{itemize}

\paragraph{MPI\_Comm\_compare函数}~{}

int MPI\_Comm\_compare(
    
    \qquad MPI\_Comm comm1,//[传入]要比较的通信子1

    \qquad MPI\_Comm comm2 //[传入]要比较的通信子2
    
    )

\begin{itemize}
    \item 功能:比较两个通信子
    \item 返回值:
    
    {
        \begin{itemize}
            \item MPI\_IDENT:两个通信子的组和上下文相同。
            \item MPI\_CONGRUENT:上下文不同、组相同。
            \item MPI\_SIMILAR:上下文不同,组的成员相同但次序不同。
            \item MPI\_UNEQUAL:都不相同。
            \item 失败:错误代码。    
        \end{itemize}
    }
    \item 备注:无。
\end{itemize}


\paragraph{MPI\_Comm\_create函数}~{}

int MPI\_Comm\_Create(
    
    \qquad MPI\_Comm comm, //[传入]源通信子

    \qquad MPI\_Group group, //[传入]定义源通信子中请求的进程子集的组

    \qquad MPI\_Comm *newcomm //[传出]新的通信子
    
    )

\begin{itemize}
    \item 功能:提取一组进程的子集,以便在单独的通信子中进行单独的多指令多数据(MIMD)计算。
    \item 返回值:返回成功时为MPI\_SUCCESS,否则为错误代码。
    \item 备注:创建新的,老的还在。group需要自己定义。
\end{itemize}

\paragraph{MPI\_Comm\_split函数}~{}

int MPIAPI MPI\_Comm\_split(

    \qquad MPI\_Comm comm,//[传入]要拆分的通信子。也就是被划分的范围

    \qquad int color,//[传入]相同的color的通信子会被划分成同一个子通信子

    \qquad int key,//[传入]新通信子中调用进程的相对等级(rank)。进程在新的通信子中按参数键的值定义的顺序排列
    
    \qquad \_Out\_ MPI\_Comm *newcomm //[传出]新的通信子

    )

\begin{itemize}
    \item 功能:用于将指定的单个通信的进程组划分为任意数量的子组。
    \item 返回值:返回成功时为MPI\_SUCCESS,否则为错误代码。
    \item 备注:将原有的通信子拆分了,新的组成老的。且子组的数量由在所有进程中指定的color数量确定。生成的通信器不重叠。
\end{itemize}

\paragraph{MPI\_Comm\_free函数}~{}

int MPIAPI MPI\_Comm\_free(
    
    \qquad MPI\_Comm *comm //[输入]指向要释放的通信子的指针

);

\begin{itemize}
    \item 功能:释放通过dup、create或split创建的通信子。
    \item 返回值:成功时返回MPI\_SUCCESS,否则返回错误代码。
    \item 备注:
    
    此操作将通信子标记为释放。句柄设置为MPI\_COMM\_NULL。任何使用此通信子的挂起操作都将正常完成。直到没有对对象的活动引用时,对象才会被释放。

    这一功能既适用于内部通信子,也适用于外部通信子。
    
    所有缓存属性的删除被回调函数以不确定的顺序调用。

\end{itemize}

\section{点到点通信函数}

\subsection{阻塞式}

阻塞式:发送或接受完数据后该rank进程才会继续执行。而且必须发送成功(但不一定接收成功)。

\subsection{MPI\_Send函数}

MPI\_Send(

    \qquad void* data,//[传入]发送缓冲区地址

    \qquad int count,//[传入]数据大小

    \qquad MPI\_Datatype datatype,//[传入]信息的数据类型

    \qquad int dest,//[传入]目标的进程编号

    \qquad int tag,//[传入]消息标记(用于区分不同类型的消息)

    \qquad MPI\_Comm send\_comm//[传入]目标的通信子
    
    )

\begin{itemize}
    \item 功能:执行标准模式发送操作,并在可以安全地再利用发送缓冲区时返回(直到缓存为空)。
    \item 返回值:成功时返回MPI\_SUCCESS,否则返回错误代码。
    \item 备注:为非本地函数,成功完成取决于是否存在匹配的接收函数。
\end{itemize}

\subsection{MPI\_Recv函数}

int MPIAPI MPI\_Recv(

    \qquad void            *buf,//[传出]接收缓冲区地址

    \qquad int             count,//[传入]接收的数据大小

    \qquad MPI\_Datatype   datatype,//[传入]信息的数据类型

    \qquad int             source,//[传入]指定来源的进程编号,若为MPI\_ANY\_SOURCE表示任意来源

    \qquad int             tag,//[传入]指定来源的消息标记,若为MPI\_ANY\_TAG表示任意标签都接受

    \qquad MPI\_Comm       recv\_comm,//[传入](接收方)通信子,需要与send中的相同。通常情况下send和recv均为MPI\_COMM\_WOLRD

    \qquad MPI\_Status      *status //[传出]接受状态,一般不使用该参数,直接赋常量MPI\_STATUS\_IGNORE即可

);

\begin{itemize}
    \item 功能:执行接收操作,并且在收到匹配的消息之前不返回(直到缓存被填充)。
    \item 返回值:成功时返回MPI\_SUCCESS,否则返回错误代码。
    \item 备注:
    
    {
        \begin{itemize}
            \item 接收消息的长度必须小于或等于接收缓冲区的长度。如果所有传入数据都不适合接收缓冲区,则此函数将返回溢出错误。
            \item 发送和接收操作之间存在不对称性。接收操作可以接受来自任意发送方的消息,但发送操作必须指定唯一的接收方。
            \item 注意防止死锁。
        \end{itemize}
    }

\end{itemize}

\paragraph{函数成功接受的必要条件}~{}

\begin{itemize}
    \item \textcolor{red}{send\_comm==recv\_comm}
    \item \textcolor{red}{send\_tag==recv\_tag}
    \item \textcolor{red}{send\_dest==recv\_rank(进程编号)}
    \item \textcolor{red}{send\_tag==recv\_tag}
\end{itemize}

\subsection{MPI\_Sendrecv合成函数}

int MPIAPI MPI\_Sendrecv(
  
    \qquad void         *sendbuf,//[传入]发送缓冲区地址
    
    \qquad int          sendcount,//[传入]数据大小

    \qquad MPI\_Datatype sendtype,//[传入]信息的数据类型

    \qquad int          dest,//[传入]目标的进程编号

    \qquad int          sendtag,//[传入]消息标记

    \qquad void         *recvbuf,//[传出]接收缓冲区地址

    \qquad int          recvcount,//[传入]接收的数据大小

    \qquad MPI\_Datatype recvtype,//[传入]指定来源的数据类型

    \qquad int          source,//[传入]指定来源(接收)的进程编号

    \qquad int          recvtag,//[传入]指定来源的消息标记

    \qquad MPI\_Comm     comm,//[传入](接收方)通信子

    \qquad MPI\_Status   *status//[传出]接受状态,同上
);

\begin{itemize}
    \item 功能:发送和接收消息。
    \item 返回值:成功时返回MPI\_SUCCESS,否则返回错误代码。
    \item 备注:send、recv、sendrecv互相兼容,sendrecv既可以接受send的数据,也可以给recv发送数据。
\end{itemize}

\subsection{MPI\_Sendrecv\_Replace合成函数}

int MPI\_Sendrecv\_replace(

    \qquad void* buffer,//[传入传出]发送和接收缓冲区的初始地址

    \qquad int count,//[传入传出]数据的大小

    \qquad MPI\_Datatype sendtype,//[传入传出]数据的类型

    \qquad int dest,//[传入]目标的进程编号rank

    \qquad int sendtag,//[传入]发送的信息的消息标记

    \qquad int source,//[传入]指定来源的进程编号

    \qquad int recvtag,//[传入]指定来源的消息标记

    \qquad MPI\_Comm comm,//[传入](接收方)通信子

    \qquad MPI\_Status*status//[传出]接受状态,同上
)

\begin{itemize}
    \item 功能:使用单个缓冲区发送和接收消息。
    \item 返回值:成功时返回MPI\_SUCCESS,否则返回错误代码。
    \item 备注:与Sendrecv相比,不同之处在于使用同一个缓冲区来接收和发送数据(因此前三个参数是一样的)。也正因为如此,效率相比于Sendrecv低下。
\end{itemize}




\begin{itemize}
    \item 功能:
    \item 参数:
    {
        \begin{itemize}
            \item 
        \end{itemize}
    }
    \item 备注:
\end{itemize}

\begin{itemize}
    \item 功能:
    \item 参数:
    {
        \begin{itemize}
            \item 
        \end{itemize}
    }
    \item 备注:
\end{itemize}












\begin{itemize}
    \item 
\end{itemize}








\end{document}%文章结束